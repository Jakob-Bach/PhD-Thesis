Die Merkmalsauswahl (\emph{feature selection}) im maschinellen Lernen zielt darauf ab, die nützlichsten Variablen in einem Datensatz für Vorhersagen zu finden.
Methoden zur Merkmalsauswahl sind aus vielfältigen Gründen allgegenwärtig:
Sie können die Vorhersagequalität verbessern, Hardwareanforderungen verringern und das Verständnis der Daten erleichtern.
Allerdings reichen bestehende Methoden zur Merkmalsauswahl in bestimmten Szenarien nicht aus, um Nutzerbedürfnisse zu befriedigen:
(1)~Manche Nutzer möchten Domänenwissen in die Merkmalsauswahl miteinbeziehen.
Beispielsweise können etablierte Gesetzmäßigkeiten oder Hypothesen aus der Domäne die Auswahl bestimmter Merkmalskombinationen unintuitiv für Nutzer machen.
Bestehende Methoden zur Merkmalsauswahl ignorieren jedoch typischerweise Domänenwissen oder unterstützen nur bestimmte Arten davon.
(2)~Mehrere, unterschiedlich zusammengesetzte Merkmalsmengen können gute Vorhersagen liefern.
Solche Alternativen können Nutzern verschiedene Erklärungen für Vorhersagen bieten.
Bestehende Methoden zur Merkmalsauswahl geben jedoch typischerweise nur eine Merkmalsmenge zurück.

In dieser Dissertation machen wir die Merkmalsauswahl Nutzer-zentrierter, indem wir Nebenbedingungen (\emph{constraints}) für die Zusammensetzung der Merkmalsmengen einführen.
Die Integration solcher Nebenbedingungen in bestehende Methoden zur Merkmalsauswahl ist herausfordernd, da Nebenbedingungen die zulässigen Merkmalsmengen beliebig einschränken können, insbesondere beim Kombinieren verschiedener Arten von Nebenbedingungen.
Weiterhin können Nebenbedingungen die Vorhersagequalität der Merkmalsmengen negativ beeinflussen.
Unsere Arbeit leistet vier wesentliche Beiträge:

(1) Wir untersuchen den Einfluss von Nebenbedingungen auf die Ergebnisse der Merkmalsauswahl.
Als Erstes formalisieren wir Merkmalsauswahl unter Nebenbedingungen als ein Optimierungsproblem.
Unsere grundlegende Problemdefinition ist unabhängig von der Methode zur Merkmalsauswahl.
Um Nebenbedingungen zu berücksichtigen, setzen wir einen Löser für Satisfiability-Modulo-Theories-(SMT)-Probleme ein.
Dieser erlaubt die Nutzung und Kombination einer großen Bandbreite von Nebenbedingungen.
Als Zweites werten wir den Einfluss von Nebenbedingungen empirisch aus.
Wir beobachten einen Zielkonflikt zwischen der Stärke der Nebenbedingungen und der Vorhersagequalität der ausgewählten Merkmale.
Dieser Effekt ist jedoch nichtlinear; insbesondere können starke Nebenbedingungen immer noch zu einer hohen Qualität der Merkmalsmenge führen.

(2) Wir nutzen Nebenbedingungen zum Ausdrücken und Vergleichen wissenschaftlicher Hypothesen in einem materialwissenschaftlichen Anwendungsfall.
Insbesondere kooperieren wir mit Domänenexperten, um die entsprechenden Nebenbedingungen zu formulieren.
Unsere Experimente zeigen, dass manche Arten der Nebenbedingungen zu unterschiedlich zusammengesetzten Merkmalsmengen mit gleichzeitig hoher Qualität führen.
Dieses Ergebnis motiviert unsere folgende Arbeit.

(3) Wir nutzen Nebenbedingungen zum Finden alternativer Merkmalsmengen.
Letztere sind Merkmalsmengen, die sich von anderen Merkmalsmengen unterscheiden und gleichzeitig eine möglichst hohe Qualität erreichen sollen.
Als Erstes formalisieren wir alternative Merkmalsauswahl mittels ganzzahliger linearer Nebenbedingungen und binären Entscheidungsvariablen.
Nutzer können die Anzahl und die Unterschiedlichkeit der Alternativen durch jeweils einen Parameter bestimmen.
Als Zweites diskutieren wir, wie sich bestehende Konzepte für die Qualität von Merkmalsmengen als Zielfunktion des Optimierungsproblems integrieren lassen.
Als Drittes untersuchen wir die Zeitkomplexität des Optimierungsproblems und zeigen, dass es bereits mit einer einfachen Zielfunktion $\mathcal{NP}$-schwer ist.
Als Viertes stellen wir Näherungsheuristiken für diese Zielfunktion vor.
Als Fünftes werten wir unsere Ansätze mit fünf Methoden zur Merkmalsauswahl aus.
Wir beobachten, dass unsere Ansätze alternative Merkmalsmengen mit hoher Qualität finden können und dass die beiden Parameter den Nutzern Kontrolle über die Qualität geben.

(4) Wir nutzen Nebenbedingungen, um kleine und alternative Merkmalsmengen für Untergruppen (\emph{subgroups}) zu finden.
Methoden zur Untergruppenerkennung (\emph{subgroup discovery}) suchen in einem Datensatz nach interessanten Bereichen, die sich prägnant beschreiben lassen, beispielsweise mittels logischer Konjunktionen über Intervalle für Merkmalswerte.
Als Erstes formalisieren wir Untergruppenerkennung als ein SMT-Optimierungsproblem.
Als Zweites formalisieren wir zwei Nutzer-zentrierte Arten von Nebenbedingungen:
(a)~Wir verkleinern die Beschreibungen von Untergruppen, indem wir die Zahl der genutzten Merkmale begrenzen.
(b)~Wir definieren das Problem des Findens alternativer Untergruppenbeschreibungen.
Letztere beschreiben eine gegebene Untergruppe mit anderen Merkmalen.
Als Drittes beweisen wir, dass die Optimierung von Untergruppen mit beiden Arten von Nebenbedingungen $\mathcal{NP}$-schwer ist.
Als Viertes beschreiben wir, wie sich die beiden Arten von Nebenbedingungen in bestehende Heuristiken zur Untergruppenerkennung integrieren lassen.
Als Fünftes werten wir SMT-Löser-basierte und heuristische Methoden zur Untergruppenerkennung empirisch aus.
Wir beobachten, dass heuristische Suchmethoden nicht nur schnell sind, sondern auch Untergruppen mit hoher Qualität liefern, selbst wenn beide Arten von Nebenbedingungen berücksichtigt werden.
