Feature selection in machine learning aims to identify the variables in a dataset that are most useful for predictions.
Feature-selection methods are ubiquitous for a variety of reasons:
They can increase prediction quality, reduce hardware requirements, and ease understanding of the data.
However, existing feature-selection methods do not satisfy user needs in certain scenarios:
(1)~Users may want to integrate domain knowledge into feature selection.
For example, established laws or hypotheses from the domain can make selecting certain feature combinations unintuitive for users.
In contrast, existing feature-selection methods typically ignore domain knowledge or only support particular types of it.
(2)~Multiple, differently composed feature sets may yield good predictions.
Such alternatives may provide users with different explanations for predictions.
In contrast, existing feature-selection methods typically yield just one feature set.

In this thesis, we make feature selection more user-centric by introducing constraints on the composition of feature sets.
Integrating such constraints into existing feature-selection methods is challenging since constraints may limit admissible feature sets arbitrarily, particularly when combining different constraint types.
In addition, constraints may negatively affect the feature sets' predictive quality.
Our contribution is fourfold:

(1) We analyze the impact of constraints on feature-selection results.
First, we formalize constrained feature selection as an optimization problem.
Our basic problem definition is independent of the feature-selection method.
To consider constraints, we propose employing a Satisfiability Modulo Theories (SMT) solver, which allows the use and combination of a wide range of constraint types.
Second, we evaluate the impact of constraints empirically.
We observe a trade-off between the strength of the constraints and the predictive quality of the selected features.
However, the effect is non-linear, i.e., strong constraints can still result in high feature-set quality.

(2) We use constraints to express and compare scientific hypotheses in a materials-science use case.
In particular, we collaborate with domain experts to formulate corresponding constraint types.
Our experiments demonstrate that some constraint types lead to differently composed feature sets with high quality, which motivates our following work.

(3) We use constraints to find alternative feature sets, i.e., feature sets that differ from others while simultaneously optimizing feature-set quality.
First, we formalize alternative feature selection with 0-1 integer linear constraints.
Users can control the number and dissimilarity of alternatives with one parameter each.
Second, we discuss how to integrate existing notions of feature-set quality as the objective of the optimization problem.
Third, we analyze the time complexity of this problem and show $\mathcal{NP}$-hardness, for a simple notion of feature-set quality already.
Fourth, we introduce heuristic approximations for the latter.
Fifth, we evaluate our approaches with five feature-selection methods.
We observe that our approaches can find high-quality alternative feature sets, and the two parameters allow users to exercise control over the quality.

(4) We use constraints to find sparse and alternative feature sets for subgroups.
Subgroup-discovery methods search interesting regions in a dataset that admit concise descriptions, e.g., a logical conjunction of bounds on feature values.
First, we formalize subgroup discovery as an SMT optimization problem.
Second, we formalize two user-centric constraint types:
(a) We make subgroup descriptions sparse by limiting the number of features used.
(b) We propose the problem of finding alternative subgroup descriptions, which use different features to describe a given subgroup.
Third, we prove $\mathcal{NP}$-hardness of optimization with either constraint type.
Fourth, we describe how to integrate both constraint types into existing heuristic subgroup-discovery methods.
Fifth, we evaluate solver-based and heuristic subgroup discovery empirically.
We observe that heuristic search methods are not only fast but also yield high-quality subgroups, even with the two constraint types.
